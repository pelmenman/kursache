\documentclass[12pt, a4paper]{article}
\usepackage[english, russian]{babel}
\usepackage[utf8]{inputenc}
\usepackage[T2A]{fontenc}
\usepackage{diagbox}
\usepackage{tikz}
\usepackage{array}
\usepackage{amsmath}
\usepackage[left=20mm, right=15mm, top=20mm, bottom=20mm]{geometry}
\nocite{*}
\usepackage[numbers]{natbib}
\makeatletter
\renewcommand*{\@biblabel}[1]{\hfill#1.}
\makeatother
\bibliographystyle{plainnat}
\usepackage{amsfonts}
\usepackage{graphicx}
\usepackage{hyperref}
\usepackage{pdfpages}

\begin{document}
%титульники
% \includepdf[]{титульник_стр1}
% \includepdf[]{титульник_стр2}

%содержание
\begin{center}
    \tableofcontents
\end{center}

\newpage

\begin{center}
    \section{Теоретическая часть}
\end{center}

\newpage

\section{Практическая часть. Постановка задачи}

\newpage

\begin{center}
    \section{Решение задачи}
\end{center}

\newpage

\begin{center}
    \section{Заключение}
\end{center}

В курсовой работе была поставлена задача реализовать алгоритмы нахождения подстроки в строке и сравнить их эффективность. В ходе работы выяснилось ***

\newpage

\begin{center}
    \section{Список использованной литературы}
    \renewcommand{\refname}{}
    \bibliography{./4/bibl}
\end{center}

\end{document}
